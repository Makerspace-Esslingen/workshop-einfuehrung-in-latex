\documentclass{article}

\usepackage[utf8]{inputenc}

\begin{document}
Der Text eines Absatzes wird von LaTeX automatisch im
Blocksatz ausgerichtet und mit Silbentrennung sowie
Zeilenumbrüchen versehen. \textbf{Fett-} und
\textit{Kursiv-}Druck sind ebenfalls möglich.
Einzelne Zeilenumbrüche im Quelltext werden im Druck ignoriert.
Zeilenumbrüche können\\
erzwungen \newline
werden.

Ein neuer Absatz beginnt automatisch, wenn im Quelltext zwei
aufeinanderfolgende Zeilenumbrüche gefunden werden. Die erste
Zeile wird dabei automatisch eingerückt.

Am Ende dieser Zeile wird ein Seitenumbruch gemacht. \pagebreak
Folgesätze, die zum selben Absatz gehören, werden benutzt, um die Zeile aufzufüllen.
Ein Seitenumbruch kann auch an Ort und Stelle erzwungen werden. \newpage
Folgesätze des aktuellen Absatzes werden dann erst auf der nächste Seite begonnen.
\end{document}
